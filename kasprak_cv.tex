\documentclass{article}
\usepackage{geometry}
\geometry{top=1in,bottom=1in,right=1in,left=1in}
\usepackage{setspace}
\pagestyle{empty}

\begin{document}

\singlespacing
\vspace{0.2cm}
\noindent Utah State University \hfill{} alan.kasprak@gmail.com\\
Watershed Sciences Department \hfill{} Tel:  435.554.8492\\
5210 Old Main Hill \hfill{} Fax: 435.797.1871\\
Logan, Utah 84322-5210

\begin{center}
	\textbf{ALAN KASPRAK - CURRICULUM VITAE}
\end{center}

\noindent \textbf{\underline{EDUCATION}}\\
\noindent \textbf{Doctor of Philosophy in Watershed Sciences}\hfill{}Entered September 2010\\
Utah State University - Logan, UT\\
Ph.D Dissertation: \textit{Investigating Braided River Response to Variations\\
in Sediment Supply Using Morphodynamic Modeling}\\

\noindent \textbf{Master of Science in Earth Sciences}\hfill{}June 2010\\
Dartmouth College - Hanover, NH\\
MS Thesis: \textit{Stream Channel and Riparian Response to Land-Use in Northern\\
New England}\\

\noindent \textbf{Bachelor of Science in Geology/Geophysics}\hfill{}May 2008\\
Boston College - Chestnut Hill, MA\\
BS Thesis: \textit{Measuring Sedimentation Rates and\\
Land-Use Change in a Dam-Influenced Lake Delta: Narraguagus River, Maine}\\

\noindent \textbf{\underline{TEACHING AND RESEARCH EXPERIENCE}}\\
\textbf{Utah State University - Logan, UT}\hfill{}9/10 - Present\\
\textit{Research Assistant}\\

\noindent \textbf{Dartmouth College - Hanover, NH}\hfill{}7/08 - 7/10\\
\textit{Teaching $\&$ Research Assistant}\\

\noindent \textbf{Boston College - Chestnut Hill, MA}\hfill{}5/07 - 8/08\\
\textit{Fluvial Systems Lab $\&$ Field Researcher}\\

\noindent \textbf{\underline{PEER-REVIEWED PUBLICATIONS}}\\
\noindent \textbf{Kasprak A}, Wheaton JM, Ashmore PE, Peirce SA. 2015. The relationship between particle path lengths and channel morphology: results from physical models of braided rivers. \textit{Journal of Geophysical Research: Earth Surface} \textbf{120:} 55-74. DOI: 10.1002/2014JF003310.\\

\noindent Hough-Snee N, \textbf{Kasprak A}, Roper BB, Meredith CS. 2014. Direct and indirect drivers of instream wood in the interior Pacific Northwest, USA: decoupling climate, vegetation, disturbance, and geomorphic setting. \textit{Riparian Ecology and Conservation} \textbf{2:} 14-34. DOI: 10.2478/remc-2014-0002.\\

\newpage
\noindent \textbf{\underline{PEER-REVIEWED PUBLICATIONS - CONTINUED}}\\
\noindent Wheaton JM, Brasington J, Darby SE, \textbf{Kasprak A}, Sear D, Vericat D. 2013. Morphodynamic signatures of braiding mechanisms as expressed through change in sediment storage in a gravel-bed river. \textit{Journal of Geophysical Research: Earth Surface} \textbf{118:} 1-21. DOI: 10.1002/jgrf.20060\\ 

\noindent \textbf{Kasprak A}, Magilligan FJ, Nislow KH, Renshaw CE, Snyder NP, Dade WB. 2013. Differentiating the relative importance of land cover change and geomorphic processes on fine sediment sequestration in a logged watershed. \textit{Geomorphology} \textbf{185:} 67-77. DOI: 10.1016/j.geomorph.2012.12.005\\

\noindent \textbf{Kasprak A}, Magilligan FJ, Nislow KH, Snyder NP. 2012. A lidar-derived evaluation of watershed-scale large woody debris sources and recruitment mechanisms: coastal Maine, USA. \textit{River Research and Applications} \textbf{28:} 1462-1476. DOI: 10.1002/rra.1532\\

\noindent \textbf{\underline{PEER-REVIEWED PUBLICATIONS IN PROGRESS}}\\
\noindent \textbf{Kasprak A}, Hough-Snee N, Beechie T, Bouwes N, Brierley GJ, Camp R, Fryirs KA, Imaki H, Jensen ML, O'Brien G, Rosgen DL, Wheaton JM. \textit{In Review}. Choosing the Right Tool for the Job: Comparing Channel Classification Frameworks \textit{Water Resources Research}.\\

\noindent \textbf{Kasprak A}, Hafen K, Wheaton JM. \textit{In Preparation}. Coming to Grips with Model Imperfection: Morphodynamic Models as Exploratory Tools for Understanding Braided River Dynamics. \textit{Journal of Geophysical Research: Earth Surface.}\\

\noindent \textbf{Kasprak A}, Hafen K, Wheaton JM. \textit{In Preparation}. The Sensitivity of Braided River Morphodynamics to Variations in Sediment Source. \textit{Geology}.\\

\noindent \textbf{\underline{SCIENTIFIC REPORTS}}\\
\textbf{Kasprak A}, Wheaton JM. 2012. Development of a rapid geomorphic assessment procedure for streams in the John Day River Watershed, Oregon. Prepared for EcoLogical Research, Providence, UT. 126 p.\\

\noindent \textbf{\underline{MEETING ABSTRACTS}}\\
\noindent \textbf{Kasprak A}, Wheaton JM, Ashmore P, Peirce S. 2013. The sensitivity of sediment path-lengths to channel morphology: results from physical models of braided rivers. \textit{EOS, Transactions, American Geophysical Union}. San Francisco, CA - December 9-13, 2013.\\

\noindent \textbf{Kasprak A}, Wheaton JM, Bouwes N, Weber NP, Trahan NC, Jordan CE. Toward a rapid synthesis of field and desktop data for classifying streams in the Pacific Northwest: guiding the sampling and management of salmonid habitat. \textit{EOS, Transactions, American Geophysical Union}. San Francisco, CA - December 3-7, 2012.\\

\noindent \textbf{Kasprak A}, Wheaton JM. Morphodynamic modeling of gravel-bed rivers: a step-length based approach. \textit{EOS, Transactions, American Geophysical Union}. San Francisco, CA - December 5-9, 2011.\\

\noindent \textbf{\underline{MEETING ABSTRACTS - CONTINUED}}\\
\noindent \textbf{Kasprak A}, Wheaton JM. Modeling gravel bed river morphodynamics using a step-length-based approach. Community Surface Dynamics Modeling System 2011 Meeting: Impact of Time and Process Scales. Boulder, CO - October 28-30, 2011.\\

\noindent \textbf{Kasprak A}, Wheaton JM. A new step-length-based morphodynamic model of gravel-bed river evolution. \textit{Abstracts with Programs}. Geological Society of America. Minneapolis, MN - October 8-12, 2011.\\

\noindent \textbf{Kasprak A}, Magilligan FJ, Nislow KH, Snyder NP. A lidar-derived evaluation of watershed-scale large woody debris sources and recruitment mechanisms: coastal Maine, USA. \textit{EOS, Transactions, American Geophysical Union}. San Francisco, CA - December 13-17, 2010.\\

\noindent \textbf{Kasprak A}, Magilligan FJ, Nislow KH, Snyder NP. Evaluating the impacts of land-use change on stream morphology in coastal Maine.\textit{EOS, Transactions, American Geophysical Union}. San Francisco, CA - December 14-18, 2009.\\

\noindent \textbf{Kasprak A}, Magilligan FJ, Nislow KH, Snyder NP. A rapid, lidar-based delineation of watershed-scale large woody debris sources. \textit{Abstracts with Programs}. Geological Society of America. Portland, OR - December 18-21, 2009.\\

\noindent \textbf{Kasprak A}, Arcone SA, Dade WB, Finnegan DC, Magilligan FJ, Renshaw CE. 2008. Using ground penetrating radar to estimate sediment accumulation in a reservoir: Ball Mountain Dam, West River, Vermont. \textit{EOS, Transactions, American Geophysical Union}. San Francisco, CA - December 15-19, 2010.\\

\noindent \textbf{Kasprak A}, Buynevich IV, Johnson EA, Snyder NP. 2008. Measuring sedimentation rates and land-use change in a dam-influenced lake delta. \textit{Abstracts with Programs}. Geological Society of America, Northeastern Section. Buffalo, NY - March 27-29, 2008. Also presented at \textit{Research Forum: Atlantic Salmon and their Ecosystems}. Orono, ME - January 8-9, 2008.\\

\noindent \textbf{\underline{COURSES TAUGHT}}\\
\textbf{\textit{CO-INSTRUCTOR}}\\
\textit{Intermountain Center for River Restoration and Rehabilitation}\\
\indent Geomorphic Change Detection: Restoration Monitoring \hfill 2011 $\&$ 2014\\
\textit{Utah State University Watershed Sciences Graduate Induction Course}\\
\indent An Introduction to Stream and Landscape Classification \hfill 2014\\

\noindent \textbf{\textit{TEACHING ASSISTANT}}\\
\textit{Utah State University - Watershed Sciences Department}\\
\indent{Watershed Sciences Graduate Induction Course \hfill 2012 \& 2013}\\
\textit{Intermountain Center for River Restoration and Rehabilitation}\\
\indent Geomorphology and Sediment Transport in Channel Design \hfill 2011\\
\textit{Dartmouth College - Earth Sciences Department}\\
\indent Introduction to Earth Science\hfill{}2008 $\&$ 2010\\
\indent Off-Campus Program (Western US Geology) \hfill{} 2009\\
\noindent \textbf{\textit{TEACHING ASSISTANT - CONTINUED}}\\
\indent Oceanography \hfill{} 2009\\
\indent Earth's Past, Present, and Future Climate \hfill{} 2009\\

\noindent \textit{\textbf{LABORATORY TEACHING ASSISTANT}}\\
\textit{Dartmouth College}\\
\indent Introduction to Earth Science Laboratory \hfill{} 2008 $\&$ 2010\\

%\newpage
%
%\begin{center}
%	\textbf{\huge{Alan Kasprak}}
%\end{center}
%\hline
%\hline
%\vspace{0.2cm}


\noindent \textbf{\underline{GUEST LECTURER}}\\
\textit{\textbf{UTAH STATE UNIVERSITY}}\\
\textit{Fluvial Hydraulics and Ecohydraulics Graduate Course}\hfill 2014\\
Introduction to Two-Dimensional Eco-Hydraulic Modeling\\
\textit{EcoLunch Brown Bag Seminar}\hfill 2012\\
Life, Landscape, and the Dynamic Nature of Physical Habitat\\

\noindent \textit{\textbf{DARTMOUTH COLLEGE}}\\
\textit{Off-Campus Program (Western U.S. Geology)}\hfill 2009\\
Sediment transport in the Grand Canyon\\
\textit{Off-Campus Program (Western U.S. Geology)}\hfill 2009\\
Ephemeral stream morphology in Death Valley\\
%\noindent \textbf{\underline{INVITED GUEST LECTURER (continued)}}\\
\textit{Geolunch Brown Bag Series}\hfill 2009\\
Anthropogenically-Driven Fluvial Geomorphology\\

%\noindent \textbf{\underline{GUEST LECTURER - CONTINUED}}\\
\noindent \textit{\textbf{BOSTON COLLEGE}}\\
\textit{Geology and Geophysics Research Seminar}\hfill{}2008\\
Measuring sedimentation rates and land-use change in a\\ dam-influenced lake delta: Narraguagus River, Maine\\

\noindent \textbf{\underline{GRANTS, AWARDS, HONORS}}\\
Utah State University\\
\indent Doctoral Dissertation Completion Award\hfill{}Grant Recipient (\$20,000)\\
National Science Foundation\\
\indent Research Grant - 'Sensitivity of Braided River Morphodynamics to Sediment Supply'\\
\indent Co-Authored with PI JM Wheaton\hfill{}Grant Recipient (\$271,000)\\
Utah State University\\
\indent Graduate Student Travel Grant\hfill{}Grant Recipient (\$300)\\
Society for Sedimentary Geology\\
\indent Graduate Student Travel Grant\hfill{}Grant Recipient (\$500)\\
Geological Society of America\\
\indent Graduate Student Research Grant\hfill{}Grant Recipient (\$1000)\\
Dartmouth College, Office of Graduate Studies\\
\indent Graduate Student Research Grant\hfill{}Grant Recipient (\$2500)\\
\noindent Geological Society of America\\
\indent Graduate Student Travel Grant\hfill{}Grant Recipient (\$500)\\
Dartmouth College, Office of Graduate Studies\\
\indent Presentation Travel Grant\hfill{}Grant Recipient (\$300)\\
\noindent \textbf{\underline{GRANTS, AWARDS, HONORS - CONTINUED}}\\
Boston College Geology $\&$ Geophysics Department\\
\indent Best Undergraduate Research Presentation\hfill{}Award Winner\\
Geological Society of America, Northeastern Section\\
\indent Undergraduate Travel Grant\hfill{}Grant Recipient (\$100)\\

\noindent \textbf{\underline{CONFERENCE SYMPOSIA CONVENED}}\\
Using predictive models to inform river management and restoration. \textit{With} Gregory Pasternack, UC Davis. \textit{EOS, Transactions, American Geophysical Union}. San Francisco, CA - December 9-13, 2013.\\

\newpage
\noindent \textbf{\underline{SIGNIFICANT FIELD EXPERIENCE}}\\
University of Western Ontario - Braided River Morphodynamics (Flume) \hfill{}2013\\
Cairngorms National Park, Scotland - Braided River Morphodynamics \hfill{}2013\\
Grand Canyon, Arizona - Fine Sediment Transport and Storage \hfill{} 2012\\
John Day River Watershed, Oregon - Linking Gemorphology and Salmonid Habitat \hfill{} 2011\\
New Mexico/Colorado - Tectonics, Glacial Geomorphology (Trip Leader) \hfill{} 2010\\
Downeast Maine - Fluvial and Riparian Habitat Surveying (Trip leader) \hfill{} 2009\\
Montana, Idaho, Eastern Washington - Glacial and Volcanic History \hfill{} 2008\\
Vermont and New Hampshire - Riparian Zone Surveying \hfill{} 2008\\
Downeast Maine - Fluvial Morphology and Sedimentation (Trip Leader) \hfill{} 2007\\
New York/Vermont/New Hampshire/Maine - Appalachian Geology \hfill{} 2007\\

\noindent \textbf{\underline{RELEVANT SKILLS}}\\
\noindent \textbf{Field:} Total station, rtkGPS, terrestrial laser scanner, acoustic doppler current profiler, autolevel, stream gauging equipment, structure-from-motion point cloud generation, data logging using iPad/Android platforms\\

\noindent \textbf{Office:} Windows, Mac OS X, Linux/Unix. Proficient in C++, MATLAB, Python, Visual Studio and Qt IDEs, ArcGIS, ENVI, Cyclone, Trimble GeoOffice, AgiSoft, Adobe Photoshop/Illustrator, Hydraulic models including HEC-RAS, MD-SWMS, Hydro2de, Delft3D, \LaTeX\\

\end{document}
